\section{Network IO}
\label{Sec::NetworkIO}

When communicating to other services in the network, sometimes low-level socket IO is inevitable.
For this reason, \libcppa provides \emph{brokers}.
A broker is an event-based actor running in the middleman that multiplexes socket IO.
It can maintain any number of acceptors and connections.
Since the broker runs in the middleman, implementations should be careful to consume as little time as possible in message handlers.
Any considerable amount work should outsourced by spawning new actors (or maintaining worker actors).

\subsection{Spawning Brokers}

Brokers are spawned using the function \lstinline^spawn_io^ and always use functor-based implementations capturing the self pointer of type \lstinline^broker*^.
For convenience, \lstinline^spawn_io_server^ can be used to spawn a new broker listening to a local port and \lstinline^spawn_io_client^ can be used to spawn a new broker that connects to given host and port or uses existing IO streams.

\begin{lstlisting}
template<spawn_options Os = no_spawn_options,
         typename F = std::function<behavior (io::broker*)>,
         typename... Ts>
actor spawn_io(F fun, Ts&&... args);

template<spawn_options Os = no_spawn_options,
         typename F = std::function<behavior (io::broker*)>,
         typename... Ts>
actor spawn_io_client(F fun,
                      io::input_stream_ptr in,
                      io::output_stream_ptr out,
                      Ts&&... args);

template<spawn_options Os = no_spawn_options,
         typename F = std::function<behavior (io::broker*)>,
         typename... Ts>
actor spawn_io_client(F fun, string host, uint16_t port, Ts&&... args);

template<spawn_options Os = no_spawn_options,
         typename F = std::function<behavior (io::broker*)>,
         typename... Ts>
actor spawn_io_server(F fun, uint16_t port, Ts&&... args);
\end{lstlisting}

\clearpage
\subsection{Broker Interface}

\begin{lstlisting}
class io::broker;
\end{lstlisting}

{\small
\begin{tabular*}{\textwidth}{m{0.51\textwidth}m{0.44\textwidth}}
  \multicolumn{2}{m{\linewidth}}{\large{\textbf{Data Types}}\vspace{3pt}} \\
  \\
  \hline
  \multicolumn{2}{m{\linewidth}}{\texttt{enum policy\_flag  \{ at\_least, at\_most, exactly \}}} \\
  \hline
  \\
  \multicolumn{2}{m{\linewidth}}{\large{\textbf{Member Functions}}\vspace{3pt}} \\
  \\
  \hline
  \lstinline^void receive_policy(^ \lstinline^connection_handle hdl,^ \lstinline^policy_flag policy,^ \lstinline^size_t buffer_size)^ & Modifies the receive policy for the connection identified by \lstinline^hdl^. This will cause the middleman to enqueue the next \lstinline^new_data_msg^ after \emph{at least}, \emph{at most}, or \emph{exactly} \lstinline^buffer_size^ bytes have been read \\
  \hline
  \lstinline^void write(connection_handle hdl,^ \lstinline^size_t num_bytes,^ \lstinline^const void* buf)^ & Sends data \\
  \hline
  \lstinline^void write(connection_handle hdl,^ \lstinline^const util::buffer& buf)^ & Sends data \\
  \hline
  \lstinline^void write(connection_handle hdl,^ \lstinline^util::buffer&& buf)^ & Sends data \\
  \hline
  \lstinline^template<typename F, typename... Ts>^ \lstinline^actor fork(F fun,^ \lstinline^connection_handle hdl, Ts&&... args)^ & Spawns a new broker that takes ownership of given connection \\
  \hline
  \lstinline^size_t num_connections()^ & Returns the number of open connections \\
  \hline
  \lstinline^connection_handle add_connection(^ \lstinline^input_stream_ptr in,^ \lstinline^output_stream_ptr out)^ & Adds a new connection from input and output stream \\
  \hline
  \lstinline^connection_handle add_connection(^ \lstinline^stream_ptr sptr)^ & Adds a new connection from an IO stream \\
  \hline
  \lstinline^connection_handle add_tcp_connection(^ \lstinline^native_socket_type tcp_sockfd)^ & Adds a new connection from a native TCP socket descriptor \\
  \hline
  \lstinline^accept_handle add_acceptor(^ \lstinline^acceptor_uptr ptr)^ & Adds a new acceptor \\
  \hline
  \lstinline^accept_handle add_tcp_acceptor(^ \lstinline^native_socket_type tcp_sockfd)^ & Adds a new acceptor from a native TCP socket descriptor \\
  \hline
\end{tabular*}
}

\clearpage
\subsection{Broker-related Message Types}

Brokers, just like any other dynamically typed actor, can receive messages of any type.
However, it also receives system messages from the middleman:

\begin{lstlisting}
struct new_connection_msg {
    io::accept_handle source;
    io::connection_handle handle;
};
\end{lstlisting}

A \lstinline^new_connection_msg^ is received whenever a new incoming connection (identified by the \lstinline^handle^ field) has been accepted for one of the broker's accept handles (identified by the \lstinline^source^ field).

\begin{lstlisting}
struct new_data_msg {
    io::connection_handle handle;
    util::buffer buf;
};
\end{lstlisting}

A \lstinline^new_data_msg^ is received whenever data on a connection is ready.
The data can be accessed as buffer object (see \ref{Appendix::Buffer}).
The amount of data, i.e., how often this message is received, can be configured using the brokers receive policy (see \ref{Sec::NetworkIO::ReceivePolicy}).

It is worth mentioning that the buffer is re-used whenever possible.
This means, as long as the broker does not create any new references to the message by copying it, the middleman will always use only a single buffer per connection.

\begin{lstlisting}
struct connection_closed_msg {
    io::connection_handle handle;
};
\end{lstlisting}

A \lstinline^connection_closed_msg^ informs the broker that one of its connections has been closed.

\begin{lstlisting}
struct acceptor_closed_msg {
    io::accept_handle handle;
};
\end{lstlisting}

A \lstinline^ acceptor_closed_msg^ informs the broker that of its acceptors has been closed.
